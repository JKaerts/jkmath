\documentclass[12pt,a4paper]{article}
\usepackage[latin1]{inputenc}
\usepackage{amsmath}
\usepackage{amsfonts}
\usepackage{amssymb}
\usepackage{graphicx}
\usepackage{longtable}
\usepackage{jkmath}
\author{Jonas Kaerts}
\title{The \texttt{jkmath} package}
\begin{document}
\maketitle

\section{The package options}
The package contains a few options with regards to subsets.
\begin{description}
    \item[subsetorder] You are a person that likes your symbols for subsets to resemble the symbols used for ordering numbers.
    The command \verb|\subset| now displays the symbol $\subseteq$ while a new command \verb|\stsubset| (for strict subsets) can be used for dispaying the symbol $\subset$.
    Similar behavior occurs with \verb|\supset| and \verb|\stsupset|.
    \item[subsetnonorder] You are a person that likes variety.
    Your symbols for subsets do not resemble the usual ordering symbols.
    the command \verb|\subset| displays the symbol $\subset$ while the symbol \verb|\stsubset| displays as $\subsetneq$.
    Same for \verb|supset| and \verb|stsupset|.
    \item[subsetnonamb] You like your notation as unambiguous as possible.
    The command \verb|\subset| displays the symbol $\subseteq$ while \verb|\stsubset| displays $\subsetneq$.
    Again similar for \verb|\supset| and \verb|\stsupset|.
\end{description}
The advantage of this approach is that you can convert a document from one style of notation to another by simply changing the package option.

There are also two options, \verb|bbsets| and \verb|bfsets| concerning the display of number systems.
They provide the following shorthands:

\begin{tabular}{llll}
    Command & Option \verb|bbsets|& Option \verb|bfsets| & Usage\\
    \verb|\N| & $\mathbb{N}$ & $\mathbf{N}$ & Natural numbers\\
    \verb|\Z| & $\mathbb{Z}$ & $\mathbf{Z}$ & Integers\\
    \verb|\Q| & $\mathbb{Q}$ & $\mathbf{Q}$ & Rational numbers\\
    \verb|\R| & $\mathbb{R}$ & $\mathbf{R}$ & Real numbers\\
    \verb|\C| & $\mathbb{C}$ & $\mathbf{C}$ & Complex numbers\\
    \verb|\F| & $\mathbb{F}$ & $\mathbf{F}$ & Fields\\
    \verb|\Aff| & $\mathbb{A}$ & $\mathbf{A}$ & Affine Space\\
    \verb|\PP| & $\mathbb{P}$ & $\mathbf{P}$ & Projective Space\\
\end{tabular}

\section{Commands with arrays}
\subsection{Systems of equations}
This package uses the \verb|array|-package to define some useful math alignment.
The first is the \verb|system| environment.
There are two new column types (\verb|e| and \verb|o|) to get the spacing around operators right.
You can then call the code

\begin{verbatim}
\begin{system}{rorer}
4x & + & 3y & 7\\
2x & - & 5y & 10
\end{system}
\end{verbatim}
to get the result
\[
\begin{system}{rorer}
4x & + & 3y & 7\\
2x & - & 5y & 10
\end{system}.
\]

This allows fine control over the alignment of a system of equations while still having the correct spacing.
Note that the column type \verb|e| automatically inserts an equality sign.

\subsection{Augmented matrices}

A second class of commands are the augmented matrices.
The environment \verb|augmentedmatrix| takes two arguments $n$ and $m$ and makes a matrix of $n+m$ columns with a vertical rule after the $n$-th column, allowing the typesetting of systems with (multiple) right hand sides in matrix form.
The code

\begin{verbatim}
\begin{augmentedmatrix}{2}{2}
1 & 2 & 3 & 4\\
5 & 6 & 7 & 8
\end{augmentedmatrix}
\end{verbatim}
has the following output:
\[
\begin{augmentedmatrix}{2}{2}
1 & 2 & 3 & 4\\
5 & 6 & 7 & 8
\end{augmentedmatrix}
\]

At the moment there are two shorthand commands \verb|apmqty| and \verb|ipmqty| which take $m=1$ and $m=n$ respectively and insert parentheses.
These are used for solving systems with one right hand side and for calculating inverse matrices.
The shorthand name is inspired by the shorthands in the \verb|physics|-package.
The code
\begin{verbatim}
\amqty{2}{1 & 2 & 3 \\ 4 & 5 & 6}
\neq
\ipmqty{3}{0 & 1 & 0 & 1 & 0 & 0\\
-1 & 0 & 2 & 0 & 1 & 0\\
0 & 0 & 3 & 0 & 0 & 1}
\end{verbatim}
produces the following output:
\[\apmqty{2}{1 & 2 & 3 \\ 4 & 5 & 6}
\neq
\ipmqty{3}{0 & 1 & 0 & 1 & 0 & 0\\
-1 & 0 & 2 & 0 & 1 & 0\\
0 & 0 & 3 & 0 & 0 & 1}\]

\section{Intervals}

I often use a script to check if my code is consistent in its use of delimters since \LaTeX\ allows you to have unmatched parentheses etc. in the text.
The commands \verb|\lbrace|, \verb|\rbrace|, \verb|\lbrack| and \verb|\rbrack| are a godsend when I both want my script to give meaningful output and I only need one delimiter (such as in the \verb|system| environment).
This package depends on \verb|mathtools| so the commands \verb|\lparen| and \verb|\rparen| for parentheses are also present.

Using these delimiter commands the package also defines four types of intervals: \verb|\oointerval|, \verb|\ccinterval|, \verb|\ocinterval| and \verb|\cointerval|.
The o and c denote whether the left or right endpoint is open or closed.
\begin{longtable}[l]{ l l p{6cm} }
&\verb|\oointerval{1, 3}| $\displaystyle\rightarrow\oointerval{1, 3}$ & interval open on both ends\\
&\verb|\ocinterval{1, 3}| $\displaystyle\rightarrow\ocinterval{1, 3}$ & interval open on left side and closed on right\\
&\verb|\cointerval{1, 3}| $\displaystyle\rightarrow\cointerval{1, 3}$ & interval closed on left side and open on right\\
&\verb|\ccinterval{1, 3}| $\displaystyle\rightarrow\ccinterval{1, 3}$ & interval closed on both sides\\
\end{longtable}

All intervals scale with their argument. You can define your own shorthands for these commands.

\section{Sets}

A general macro for denoting sets is \verb|\set| which automatically places scalable braces around the argument.
A scalable version of \verb|\mid|, called \verb|\where|, is also included.
This makes sure the (readable) code
\begin{verbatim}
\set{x\in\mathbb{R} \where \frac{3}{4}x + 5 = 0}
\end{verbatim}
will give the following result:
\[\set{x\in\mathbb{R} \where \frac{3}{4}x + 5 = 0}.\]

A second macro is \verb|\restr| for denoting restritions of functions to subsets of their domain.
Simple usage is \verb|\restr{f}_U| which displays $\restr{f}_U$.

\section{Norms}

If the user has not defined a macro \verb|\norm| the package defines such a command. On this macro we build a few others.

\begin{longtable}[l]{ l l p{6cm} }
&\verb|\norm{a}| $\displaystyle\rightarrow\norm{a}$ & Norm of vector (scales with argument)\\
&\verb|\lpnorm{f}| $\displaystyle\rightarrow\lpnorm{f}$ & $L_p$-norm of a function (subscript defaults to $2$)\\
&\verb|\lpnorm[3]{f}| $\displaystyle\rightarrow\lpnorm[3]{f}$ & Optional argument for other subscript\\
&\verb|\supnorm{f}| $\displaystyle\rightarrow\supnorm{f}$ & Supremum norm\\
\end{longtable}
Changing the default subscript $2$ can be done with the function \verb|\setlpdefault{}| which takes as its argument the new default subscript.

\section{Combinatorics}

Using \verb|\genfrac| from \verb|amsmath| the package defines two commands for Stirling numbers.

\begin{longtable}[l]{ l l p{6cm} }
&\verb|\stirlingfirstkind{n}{k}| $\displaystyle\rightarrow\stirlingfirstkind{n}{k}$ & Stirling number of first kind\\
&\verb|\stirlingsecondkind{n}{k}| $\displaystyle\rightarrow\stirlingsecondkind{n}{k}$ & Stirling number of second kind\\
\end{longtable}
Shorthands for these two commands have yet to be defined.

\section{Number Theory}
Two commands (with identical results) \verb|\legendre| and \verb|\jacobi| are defined (again using \verb|\genfrac|) to typeset Legendre symbols and Jacobi symbols.
The output is identical but their name differs to make the code more readable.
\begin{longtable}[l]{ l l p{6cm} }
&\verb|\legendre{a}{p_1}| $\displaystyle\rightarrow\legendre{a}{p_1}$ & Legendre symbol\\
&\verb|\jacobi{a}{n}| $\displaystyle\rightarrow\jacobi{a}{n}$ & Jacobi symbol\\
\end{longtable}

\section{Names of mathematicians}
This section describes three simple commands \verb|\mobius|, \verb|\cech| and \verb|\erdos| so you can mention \mobius, \cech\ and \erdos\ without any pain.
\end{document}